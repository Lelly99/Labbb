\documentclass[10pt,a4paper]{article}
\usepackage[utf8]{inputenc}
\usepackage[italian]{babel}
\usepackage{amsmath}
\usepackage{amsfonts}
\usepackage{amssymb}
\usepackage{graphicx}
\usepackage[left=2cm,right=2cm,top=2cm,bottom=2cm]{geometry}
\newcommand{\rem}[1]{[\emph{#1}]}
\newcommand{\exn}{\phantom{xxx}}

\author{Gruppo 1x.By \\ Mario Rossi, Anna Bianchi \rem{non dimenticate i nomi}}
\title{Es01B: Uso dello strumento Analog Discovery 2.}
\begin{document}
\date{23 ottembre 2150}
\maketitle

\setcounter{section}{1}

\section{Utilizzo del canale di alimentazione della funzione voltmetro}

\subsection*{2.d Accensione diodo}

La tensione di alimentazione \`e stata variata nell intervallo tra $1\,\mathrm{V}$ e $5\,\mathrm{V}$

La luminosit\`a varia in maniera proporzionale alla differenza di potenziale applicata una volta superata la tensione di accensione, differente in base al colore. \\
Led Rosso: $1.55\,\mathrm{V}$ \pm $0.05\,\mathrm{V}$ \\
Led Giallo: $1.65\,\mathrm{V}$ \pm $0.05\,\mathrm{V}$ \\
Led Verde: $2.0\,\mathrm{V}$ \pm $0.1\,\mathrm{V}$ \\
Led Bianco: $2.30\,\mathrm{V}$ \pm $0.05\,\mathrm{V}$ \\
Led Rosso: $2.35\,\mathrm{V}$ \pm $0.05\,\mathrm{V}$ \\
\vspace{0.5cm}

\subsection*{2.e Misura tensione}
Utilizzando il multimetro si misura la resistenza R1 = $218\,\mathrm{Ohm}$ \pm $2\,\mathrm{Ohm}$ e la tensione ai capi dei diodi Rosso, Blu e Verde e si ottiene:

\begin{table}[h]
\centering
\begin{tabular}{|c|c|c|c|c|c|}
\hline 
V+& $\sigma$ V+  & VD & $\sigma$ VD & I(R1)  & $\sigma$ I(R1) \\
\hline 
1.999 & 0.001 & 1.805 & 0.002 & 0.00 & 0.01 \\
2.49 & 0.01 & 1.892 & 0.002 & 0.00 & 0.01 \\
2.99 & 0.01 & 1.946 & 0.002 & 0.00 & 0.01 \\
3.48 & 0.01 & 1.988 & 0.002 & 0.01 & 0.01 \\
3.99 & 0.01 & 2.02 & 0.01 & 0.01 & 0.01 \\
\hline 
\end{tabular} 
\caption{(2.e) Tensione e corrente ai capi del diodo Rosso. Tutte le tensioni in V e le correnti in A.\label{t:par1}}
\end{table}

\begin{table}[h]
\centering
\begin{tabular}{|c|c|c|c|c|c|}
\hline 
V+& $\sigma$ V+  & VD & $\sigma$ VD & I(R1)  & $\sigma$ I(R1) \\
\hline 
1.998 & 0.001 & 1.997 & 0.002 & 0.00 & 0.01 \\
2.49 & 0.01 & 2.31 & 0.01 & 0.00 & 0.01 \\
2.99 & 0.01 & 2.44 & 0.01 & 0.00 & 0.01 \\
3.49 & 0.01 & 2.55 & 0.01 & 0.00 & 0.01 \\
4.00 & 0.01 & 2.65 & 0.01 & 0.01 & 0.01 \\
\hline 
\end{tabular} 
\caption{(2.e) Tensione e corrente ai capi del diodo Verde. Tutte le tensioni in V e le correnti in A.\label{t:par1}}
\end{table}

\begin{table}[h]
\centering
\begin{tabular}{|c|c|c|c|c|c|}
\hline 
V+& $\sigma$ V+  & VD & $\sigma$ VD & I(R1)  & $\sigma$ I(R1) \\
\hline 
2.50 & 0.01 & 2.48 & 0.01 & 0.00 & 0.01 \\
3.00 & 0.01 & 2.63 & 0.01 & 0.00 & 0.01 \\
3.49 & 0.01 & 2.70 & 0.01 & 0.00 & 0.01 \\
4.00 & 0.01 & 2.76 & 0.01 & 0.01 & 0.01 \\
\hline 
\end{tabular} 
\caption{(2.e) Tensione e corrente ai capi del diodo Blu. Tutte le tensioni in V e le correnti in A.\label{t:par1}}
\end{table}

%=======================

\section{Oscilloscopio}

\subsection*{3.e Uso del trigger}

\exn 
\par
\vspace{0.5cm}
\framebox(400,30){Inserire commento sulle prove effettuate }

\begin{figure}[h]
\centering
%\includegraphics[scale=0.4]{part1.pdf}
\framebox(400,50){ (3.e) Inserire lo screenshot dell'oscilloscopio. }
\caption{(3.e) Relazione tra trigger e segnale}
\end{figure}


\subsection*{3.f Misura tensione massima ai capi del diodo}
\par 
La tensione massima ai capi del diodo misurata con i cursori risulta essere $V_{\mathrm{MAX}}= ( \exn \pm \exn ) \,\mathrm{V}$. La funzione di misura automatica fornisce il valore $V_{\mathrm{AUTO}}= xx \,\mathrm{V}$

\vspace{0.5cm} 
\framebox(400,30){Inserire commento sulla accuratezza della misura.}






\section{Partitore}

\subsection*{4.b Partitore con resistenze da 1k}


Si realizza un partitore con resistenze da $1 \,\mathrm{k}\Omega$. Valori misurati con il multimetro: R1=$993 \pm $8 Ohm, R2=$995 \pm $8 Ohm.


\begin{table}[h]
\centering
\begin{tabular}{|c|c|c|c|c|c|}
\hline 
VIN& $\sigma$ VIN  &VOUT	 & $\sigma$ VOUT& VOUT/VIN & $\sigma$ VOUT/VIN \\
\hline 
0.999 & 0.005 & 0.500 & 0.002 & 0.500 & 0.001 \\
1.995 & 0.005 & 0.999 & 0.005 & 0.501 & 0.001 \\
2.99 & 0.02 & 1.500 & 0.007 & 0.50 & 0.02 \\
3.99 & 0.02 & 2.00 & 0.01 & 0.50 & 0.03 \\
4.98 & 0.03 & 2.50 & 0.01 & 0.50 & 0.03 \\
\hline 
\end{tabular} 
\caption{(4.b) Partitore di tensione con resistenze da circa 1k. Tutte le tensioni in V.\label{t:par1}}
\end{table}

I valori misurati sono coerenti con quelli attesi in quanto il coefficiente V(out)/V(in) è proporzionale a R2/(R1+R2) = $0.50 \pm $0.02 Ohm.


\subsection*{4.d Partitore con resistenze da circa 1M}
\par
Si realizza un partitore con resistenze da $1 \,\mathrm{M}\Omega$. Valori misurati con il multimetro: R1=$1.007 \pm $0.008 \,\mathrm{M}\Omega$, R2=$1.010 \pm $0.08 \,\mathrm{M}\Omega$.


\begin{table}[h]
\centering
\begin{tabular}{|c|c|c|c|c|c|}
\hline 
VIN& $\sigma$ VIN  &VOUT	 & $\sigma$ VOUT& VOUT/VIN & $\sigma$ VOUT/VIN \\
\hline 
0.999 & 0.005 & 0.00 & 0.01 & 0.00 & 0.01 \\
1.996 & 0.005 & 0.00 & 0.001 & 0.00 & 0.01 \\
2.99 & 0.02 & 0.00 & 0.01 & 0.00 & 0.01 \\
3.99 & 0.02 & 0.001 & 0.005 & 0.00 & 0.02 \\
4.99 & 0.03 & 0.001 & 0.005 & 0.00 & 0.03 \\
\hline 
\end{tabular} 
\caption{(4.d) Partitore di tensione con resistenze da circa 1M. Tutte le tensioni in V.\label{t:par2}}
\end{table}

I valori misurati non rispecchiano l'andamento effettivo in quanto il partitore impiegato sfrutta resistenze dai valori prossimi a quella interna del multimetro ($10 MOhm).



\subsection*{4.e Resistenza di ingresso del multimetro}
Usando il modello mostrato nella scheda si ottiene
\[ \frac{R_1}{R_T} =  \frac{V_{IN}}{V_{OUT}} - (1 +  \frac{R_1}{R_2} )
\]

Con i dati con resistenze da 1k si ottiene
\[ R_1/R_{IN} =  $0.001 \pm $0.001  \rightarrow  R_{IN} > \1 k\Omega
\]


Con i dati con resistenze da 1M si ottiene
\[ R_1/R_{IN} = $0.10  \pm  &0.02   \rightarrow  R_{IN} = (\10 \pm  \2)  M\Omega
\]

\framebox(400,30){Inserire commento sulla sensibilit\`a sperimentale della misura.} 




\section{Misure di tempo e frequenza}

\subsection*{5.e Misure di frequenza}
Misure con onda sinusoidale

\begin{table}[h]
\centering
\begin{tabular}{|c|c|c|c|c|c|}
\hline 
Periodo T (s)& $\sigma$ T (s)  &Frequenza f (Hz) & $\sigma$ f (Hz) & Misura oscilloscopio (Hz) & Differenza (Hz)\\
\hline 
\exn & \exn & \exn & \exn & \exn &\exn \\
\exn & \exn & \exn & \exn & \exn &\exn \\
\exn & \exn & \exn & \exn & \exn &\exn \\
\exn & \exn & \exn & \exn & \exn &\exn \\
\hline 
\end{tabular} 
\caption{(5.e) Misura di frequenza di onde sinusoidali  e confronto con misurazione interna dell'oscilloscopio }
\end{table}


\section{Conclusioni e commenti finali}
\framebox(400,30){Inserire eventuali commenti e conclusioni finali}

\section*{Dichiarazione}
I firmatari di questa relazione dichiarano che il contenuto della relazione \`e originale, con misure effettuate dai membri del gruppo, e che tutti i firmatari hanno contribuito alla elaborazione della relazione stessa.

\end{document}
